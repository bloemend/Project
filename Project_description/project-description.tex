\documentclass[a4paper,11pt]{article}
\usepackage{graphicx}
\usepackage{times}
\usepackage[utf8]{inputenc}
\topmargin=5.mm
\oddsidemargin=0.mm
\evensidemargin=0.mm
\textheight=220.mm 
\textwidth=170.mm
\parindent 0pt
\parskip .5em
\renewcommand{\arraystretch}{1.5}

\begin{document}
\sffamily
\begin{titlepage}
\begin{center}
\textsc{NORWEGIAN UNIVERSITY OF SCIENCE AND TECHNOLOGY\\
FACULTY OF  INFORMATION TECHNOLOGY, MATHEMATICS AND ELECTRICAL ENGINEERING} \\
\vspace{0.5cm} 
\includegraphics[scale=0.5]{NTNU-logo} \\
\vspace{0.5cm}
{\Huge{PROJECT ASSIGNMENT}}
\vspace{0.5cm}
\end{center}

\begin{tabular}{@{}p{5cm}l}
Student's name:		& Ida Malene Øveråsen og Esther Bloemendaal\\
Course: 		& TTM4531, specialization project \\
Project title: 		& Privacy on Facebook and Interdependency \\
Project description: 	& \\
\end{tabular}
\\
\\
Interdependency is a reciprocal relation between two or more decision-making entities, whose actions have consequences for each other. Interdependency is a very important issue when it comes to social networks, since your privacy is affected by the privacy decision of others. This project will be directed towards the interdependent privacy issues on Facebook.
\paragraph{}
Since Facebook came out in 2006, there has been a major change in privacy and security settings. At the same time Facebook's features have been significantly upgraded (e.g., Apps), and the platform itself has expanded to several different platforms (e.g., iOS and Android). Owing to this development, the complexity of privacy-related issues has made the originally embedded privacy requirements inadequate. We are going to map and analyze this development to see how privacy settings has changed over time. We will also look at human behaviour with regard to Facebook privacy. How this affects people when it comes to, for example, personal life and future job prospects, and to what degree people are aware of the unanticipated consequences the use of Facebook can bring.
\paragraph{}
In order to carry out the behavioural research we will use Amazon Mechanical Turk, enabling us to reach a wide audience. Amazon Mechanical Turk is a marketplace for work that requires human intelligence, and works well for conducting surveys.
The key benefits of Amazon Mechanical Turk when you are conducting behavioural research is that it provides one of the largest subject pools, with both diversity and low cost. By using the results of the survey we will look into what kind of privacy settings different types of people value and map their awareness when it comes to the importance of different privacy settings. 


\vspace{0.2cm}
\begin{tabular}{@{}p{5cm}l}
Department:		& Department of Telematics \\
Supervisor:		& Gergely Biczók \\
Responsible professor: 	& Jan Audestad \\
\end{tabular}

\end{titlepage}
\end{document}
