\chapter{The History of Facebook}
\label{chp:facebookhistory} 

\section{}\label{sec:first_section}
\paragraph{}
Since Facebook was introduced to the public in 2006, it has grown to be the largest online social network (OSN) in the world. OSNs have a peer-to-peer architecture, and therefore makes it easy for members to initiate communication with whom they want, given that they are also connected to the network. OSNs also enables the possibility for people to easily publish and retrieve information about subjects of interest \cite{DPBook}. The growth of Facebook has made it necessary to introduce new ways to manage privacy and ensure a secure online environment. When it comes to OSNs the privacy embedded in the program/app etc. is not enough to ensure such an environment, due to the interdependent privacy issues. Your privacy is to a large extent affected by the privacy decision of others. In this chapter we are going to look into the history of Facebook’s privacy settings, and map the development from 2006 until today. 

\begin{figure}[h!]
\centering
\includegraphics[width=0.2\textwidth]{facebook-icon.png}
\caption{The Facebook Icon}
\end{figure}

\paragraph{}
Facebook has evolved from being a networking site for students attending Harvard to becoming a global phenomenon. Facebook's user interface has gone through several changes over the years, which has brought both joy and frustration to the users. When these changes have been made, there has also been adjustments to the default privacy settings as well \cite{EvoPriv2}. At the beginning, in 2005, when Facebook first was applied outside of Harvard University, the users personal information was only accessible to a users Facebook friends and to people connected to the same network on Facebook \cite{EvoPriv}. 


To emphasize the changes made to the default privacy settings, we have put the largest changes in the table below. 
\begin{center}
    \begin{tabular}{ | l | p{9cm} |}
    \hline
    \textbf{Year} & \textbf{Default Privacy Settings} \\ 
    \hline
    2005 & Your personal information is only accessible for 			Thefacebook users that are members of at least one of the 			groups you have specified in your privacy settings.\\ 
    \hline
    2006 & Control over privacy settings is given to the user. By 		default, the only thing that is available for everyone to see 		is information about school and specified local area. \\ 
    \hline
    2007 & Profile information on Facebook is accessible to the 		Facebook users who is a member of at least one of the networks 		you have given access to in your privacy settings (e.g., 			schools, friends, friends of friends). Name, name of school and 	profile picture (thumbnail) is available in search results by 		default.\\
    \hline
    November 2009 & You are given freedom to share what to want 		with whom you want to share it with by editing the privacy 			settings. Sharing is divided into different groups, e.g. 			"everyone", "friends", "friends of friends". Certain types of		information is shared with "everyone" by default. Information 		set to "everyone" is information that is public and available 		not only to everyone on Facebook, but also to people not logged 	into Facebook. Every user on Facebook should review their 			privacy settings, and change the default settings if necessary.
	\\
	\hline
    December 2009 & Public information is name, profile picture, 		gender, friend list, pages you are a fan of, where you live and 	networks you are connected to. Since this information is 			considered public, there is no privacy settings connected to 		it. The only thing you can do to prevent people from seeing 		this information is to edit your search privacy settings. This 		way you you make it harder for people to find the 					information through searching. \\
    \hline
     \\
    \hline
    \end{tabular}
\end{center}



\markboth{}{}