\chapter{Related Work}
\label{chp:relatedwork} 


\section{Online Social Networks}
\paragraph{}

Online social networks (OSNs)  
-grown
-more popular
-more complex
hundrers of millions active users
understanding of user behaviour
-user activity increased 
-http://ieeexplore.ieee.org/stamp/stamp.jsp?tp=&arnumber=5578911

"A social networking service is a platform to build social networks or social relations among people who, for example, share interests, activities, backgrounds, or real-life connections." - http://en.wikipedia.org/wiki/Online_social_network#Typical_features


A user is often represented with a profile on OSNs. 

Since Facebook was introduced to the public in 2006, it has grown to be the largest online social network (OSN) in the world. OSNs have a peer-to-peer architecture, and therefore makes it easy for members to initiate communication with whom they want, given that they are also connected to the network. OSNs also enables the possibility for people to easily publish and retrieve information about subjects of interest \cite{DPBook}.


The growth of Facebook has made it necessary to introduce new ways to manage privacy and ensure a secure online environment. When it comes to OSNs the privacy embedded in the program/app etc. is not enough to ensure such an environment, due to the interdependent privacy issues. Your privacy is to a large extent affected by the privacy decision of others.

\begin{figure}[h!]
\centering
\includegraphics[width=0.2\textwidth]{facebook-icon.png}
\caption{The Facebook Icon}
\end{figure}


\section{Interdependent Privacy}


\section{The History of Facebook}
When Mark Zuckerberg enrolled at Harvard in 2002, he had decided to major in psychology  “I just think people are the most interesting thing—other people,” he said. “What it comes down to, for me, is that people want to do what will make them happy, but in order to understand that they really have to understand their world and what is going on around them” \cite{MeMedia}. He showed an interrest and passion to connect people together and crate Harvard more open. 

\paragraph{}
It all started in October 2003 when the Harvard sophomore Mark Zuckerberg and three of his classmates created the web page facesmash. Zuckerberg hacked into the administrative database to extract the ID photos of all the students of the different houses. The web page presented two and two photos creating a “hot or not” game for his fellow students. The votes were counted and created a top-ten list of the cutest poeple in each house. Within the first hour facesmash had 450 visitors and 22 000 photo-views. After numerous complaints from professors and fellow students Harvard administration shut down Zuckerbergs Internet connection after a few days. Harvard charged Zuckerberg for violating individual privacy, violating privacy and breach of security for stealing the photos. Zuckerberg agreed to take the web page down and got away with just a warning.

\paragraph{}
After facesmash Zuckerburg was known around campus as a programming prodigy. Harvard seniors Tyler and Cameron Winklevoss and Divya Narendra had since 2002 been working on a social networking page - HarvardConnection, where students could create a profile, and though that share some personal information and post pictures and share this with large and small communities that one are part off. They wanted Zuckerbergs help to finalize their project so that the page could be up and running before they graduated. Zuckerberg agreed to help at the same time as presuing his own projects. Harvard offers a class directory to all freshmans, this directory is also known as the "facebook". This "facebook" contains a picture of all the students, name, date of birth, home town and high school. Harvard's plan was to eventually get this online, so Zuckerburg decided to to the job himself. He wanted to create a page where people signed up and created their own profiles, and in that way could post some personal information about themselves, and have control over what was posted. After ten days of intensive work Zuckerberg almost finished the cite. The cite was kept simple and intuitive,and everybody with and Harvard email address could create a profile. The profile consisted of a profile picture name and some personla information such as taste in books, music, films and quotes. Users could link to their friend's profiles and by using a "poke" button let others know that you have visited their profile.  Thefacebook when public February 4, 2004 




\cite{FacebookHistory}.


\section{Amazon Mechanical Turk}


