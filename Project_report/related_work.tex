\chapter{Related Work}
\label{chp:relatedwork} 


\section{Social Network Services}
\paragraph{}
A social network service (SNS), is a platform used to establish social networks of different people. These people often share a common interest or activity \cite{SNS}. 
\paragraph{}
Online social networks (OSNs) is a large part of the social network services. From online social networks was first introduced until today, the popularity and complexity has grown drastically, with a hundreds of millions active users \cite{OSN}. OSNs have a peer-to-peer architecture, and therefore makes it easy for members to initiate communication with whom they want, given that they are also connected to the network. OSNs also enables the possibility for people to easily publish and retrieve information about subjects of interest \cite{DPBook}. 

The internet has caused the creation of several information sharing systems. Among these systems are the Web and OSNs. As mentioned before, the popularity of OSNs has grown drastically, and have become among the most popular sites on the Web. With this change, there has also been a change in what is centralized and in focus. The Web is to a large extent organized around content, while OSNs on the other hand are organized around users. This change has lead to the importance of understanding user behaviour. A user is often represented with a profile on OSNs. To obtain a profile the user, in most cases, must register the site. When a user is given a profile, it is normal for the user to provide information about themselves. This information could for example be date of birth, home town, sex, name (or pseudonym) and maybe a profile picture. The social network is formed when users start connecting with each other. The reason for these connections are numerous; real-life friends, real-life acquaintances, colleagues, share an interest/activity or if you are interested in the information contributed by the other user \cite{OSNpaper}. 

\paragraph{}
Since Facebook was introduced to the public in 2006, it has grown to be the largest online social network (OSN) in the world. The growth of Facebook has made it necessary to introduce new ways to manage privacy and ensure a secure online environment. The privacy embedded in the program/app etc. is not enough to ensure such an environment, due to the interdependent privacy issues. Your privacy is to a large extent affected by the privacy decision of others. We will elaborate these privacy issues later in our report. 


\section{Interdependent Privacy}

\section{The History of Facebook}
When Mark Zuckerberg enrolled at Harvard in 2002, he had decided to major in psychology  “I just think people are the most interesting thing—other people,” he said. “What it comes down to, for me, is that people want to do what will make them happy, but in order to understand that they really have to understand their world and what is going on around them” \cite{MeMedia}. He showed an interrest and passion to connect people together and crate Harvard more open. 

\begin{figure}[h!]
\centering
\includegraphics[width=0.2\textwidth]{facebook-icon.png}
\caption{The Facebook Icon}
\end{figure}

\paragraph{}
It all started in October 2003 when the Harvard sophomore Mark Zuckerberg and three of his classmates created the web page facesmash. Zuckerberg hacked into the administrative database to extract the ID photos of all the students of the different houses. The web page presented two and two photos creating a “hot or not” game for his fellow students. The votes were counted and created a top-ten list of the cutest poeple in each house. Within the first hour facesmash had 450 visitors and 22 000 photo-views. After numerous complaints from professors and fellow students Harvard administration shut down Zuckerbergs Internet connection after a few days. Harvard charged Zuckerberg for violating individual privacy, violating privacy and breach of security for stealing the photos. Zuckerberg agreed to take the web page down and got away with just a warning.

\paragraph{}
After facesmash Zuckerburg was known around campus as a programming prodigy. Harvard seniors Tyler and Cameron Winklevoss and Divya Narendra had since 2002 been working on a social networking page - HarvardConnection, where students could create a profile, and though that share some personal information and post pictures and share this with large and small communities that one are part off. They wanted Zuckerbergs help to finalize their project so that the page could be up and running before they graduated. Zuckerberg agreed to help at the same time as presuing his own projects. Harvard offers a class directory to all freshman's, this directory is also known as the "facebook". This "facebook" contains a picture of all the students, name, date of birth, home town and high school. Harvard's plan was to eventually get this online, so Zuckerburg decided to to the job himself. He wanted to create a page where people signed up and created their own profiles, and in that way could post some personal information about themselves, and have control over what was posted. After ten days of intensive work Zuckerberg almost finished the cite. The cite was kept simple and intuitive,and everybody with and Harvard email address could create a profile. The profile consisted of a profile picture name and some personal information such as taste in books, music, films and quotes. Users could link to their friend's profiles and by using a "poke" button let others know that you have visited their profile. Thefacebook went public February 4 2004, and to get the word spread they sent it out on the Kirkland house mailing list, that contained over 300 students. The word spread to the other houses and within twenty-four hours close to fifteen hundred people ha registered. “I think it’s kind of silly that it would take the university a couple of years to get around to it,” he said. “I can do it better than they can, and I can do it in a week.” 

\paragraph{}
There was already similar pages out there, like Friendster and myspace.com. Especially on myspace.com people played roles, giving themselves out to be someone else. Teenage girls pretending to be older and grown men giving themselves out to be young girls. There is nowhere to validate that the person really is who thy give themselves out to be. With Thefacebook.com you know that most were student since you have to sign up with an valid Harvard email address. This made it easier to post more personal information like cell-phone number, home address and even sexual orientation. The concern was not about security but more about wasting time, it became and addictive pleasure. 

\paragraph{}
It didn't take long before Mark Zuckerberg began to receive emails from other colleges, requestion to get Thefacebook at their schools. It was an important concern of Zuckerberg to keep the site's intimacy. When Thefacebook expanded to Colombia, Yale and Stanford, students were only able to search and see people from their respective college. Only with permission from a student from an other college could you ass the person to your friend list. This is a key factor to Facebook's success. Zuckerberg wanted people to post personal information and create a more open school community.

\paragraph{}
In June 2004, when the school year was over, Thefacebook had expanded to over forty schools, with a 150 000 users. Mark Zuckerberg had received many offers from venture capitalist wanting to invest in web page. 



\cite{FacebookHistory}.


\section{Amazon Mechanical Turk}
\paragraph{}







