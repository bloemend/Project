\chapter{Related Work}
\label{chp:relatedwork} 


\section{Social Network Services}
\paragraph{}
A social network service (SNS), is a platform used to establish social networks of different people. These people often share a common interest or activity \cite{SNS}. Online social networks (OSNs) is a large part of the social network services. From online social networks was first introduced until today, the popularity and complexity has grown drastically, with a hundreds of millions active users \cite{OSN}. OSNs have a peer-to-peer architecture, and therefore makes it easy for members to initiate communication with whom they want, given that they are also connected to the network. OSNs also enables the possibility for people to easily publish and retrieve information about subjects of interest \cite{DPBook}. The internet has caused the creation of several information sharing systems \cite{OSNpaper}. Among these systems are the Web and OSNs. As mentioned before, the popularity of OSNs has grown drastically, and have become among the most popular sites on the Web. With this change, there has also been a change in what is centralized and in focus. The Web is to a large extent organized around content, while OSNs on the other hand are organized around users. This change has lead to the importance of understanding user behaviour. You can say that the expansion of OSNs has lead to a shift in how context is exchanged over the Web. End users are no longer just content consumers, but now also required to be content creators and managers \cite{expectations}.

\paragraph{}
A user is often represented with a profile on OSNs. To obtain a profile the user, in most cases, must register the site. When a user is given a profile, it is normal for the user to provide information about themselves. This information could for example be date of birth, home town, sex, name (or pseudonym) and maybe a profile picture. The social network is formed when users start connecting with each other. The reason for these connections are numerous; real-life friends, real-life acquaintances, colleagues, share an interest/activity or if you are interested in the information contributed by the other user. 

\paragraph{}
Since Facebook was introduced to the public in 2006, it has grown to be the largest online social network (OSN) in the world. The growth of Facebook has made it necessary to introduce new ways to manage privacy and ensure a secure online environment. The privacy embedded in the program/app etc. is not enough to ensure such an environment, due to the interdependent privacy issues. Your privacy is to a large extent affected by the privacy decision of others. 


\section{The History of Facebook}
\label{sec:facebookhistory}
When Mark Zuckerberg enrolled at Harvard in 2002, he had decided to major in psychology. “I just think people are the most interesting thing—other people,” he said. “What it comes down to, for me, is that people want to do what will make them happy, but in order to understand that, they really have to understand their world and what is going on around them” \cite{MeMedia}. He showed an interest and passion to connect people together and create Harvard more open. 

\begin{figure}[h!]
\centering
\includegraphics[width=0.2\textwidth]{facebook-icon.png}
\caption[Facebook icon]{\textbf{The Facebook icon} as we know it today.}
\end{figure}

\paragraph{}
It all started in October 2003 when the Harvard sophomore Mark Zuckerberg and three of his classmates created the web page Facemash. Zuckerberg hacked into the administrative database to extract the ID photos of all the students of the different houses. The web page presented two and two photos creating a “hot or not” game for his fellow students. The votes were counted and created a top-ten list of the cutest people in each house. Within the first hour Facemash had 450 visitors and 22 000 votes. After numerous complaints from professors and fellow students, Harvard administration shut down Zuckerberg's Internet connection after a few days. Harvard charged Zuckerberg for violating individual privacy, violating privacy and breach of security for stealing the photos. Zuckerberg agreed to take the web page down and got away with just a warning.

\paragraph{}
After Facemash, Zuckerburg was known around campus as a programming prodigy. Harvard seniors, Tyler and Cameron Winklevoss and Divya Narendra had since 2002 been working on a social networking page, called HarvardConnection. This was a page where students could create a profile, and through that share some personal information and post pictures and share this with large and small communities that one could be part off. They wanted Zuckerberg's help to finalize their project so that the page could be up and running before they graduated. Zuckerberg agreed to help at the same time as pursuing his own projects. Harvard offers a class directory to all freshmen, this directory is also known as the "Facebook". This "Facebook" contains a picture of all the students, name, date of birth, home town and high school. The purpose of the "Facebook" was that the freshmen could get to know each other. Harvard's plan was to eventually get this online. Since Harvard had not gotten to it yet,  Zuckerburg decided to do the job himself. He wanted to create a page where people signed up and created their own profiles, and in that way could post some personal information about themselves, and have control over what was posted. After ten days of intensive work, Zuckerberg almost finished the site. The site was kept simple and intuitive, and everybody with a Harvard e-mail address could create a profile. The profile consisted of a profile picture, name and some personal information such as taste in books, music, films and favourite quotes. Users could link to their friend's profiles and by using a "poke" button let others know that you have visited their profile. Thefacebook went public February 4, 2004, and to get the word spread they sent it out on the Kirkland house mailing list, that contained over 300 students. It did not take long until the other houses heard and within twenty-four hours, close to fifteen hundred people ha registered. “I think it’s kind of silly that it would take the university a couple of years to get around to it,” he said. “I can do it better than they can, and I can do it in a week.” \cite{MeMedia}. Later the same year the three founders of HarvardConnection, now called ConnectU, filed a lawsuit against Zuckerburg. Stating that he broke their oral contract, stole their idea, and delayed working on their site to be able to finish his own site, Thefacebook, first. Zuckerburg denied doing anything wrong, and stated that he had proof that he did not steal the idea from the HarvardConnection. Just a few months later Facebook filed a countersuit. Facebook accused ConnectU with defamation. The case went on for years. In 2011 the Winklevoss brothers dropped the lawsuit and accepted a $ 65 $ million settlement \cite{droppLawsuit}.

\paragraph{}
There was already similar pages out there, like Friendster and myspace.com. Especially on myspace.com people played roles, giving themselves out to be someone else. Teenage girls pretending to be older and grown men giving themselves out to be young girls. There is nowhere to validate that the person really is who they give themselves out to be. This limits to what extent people posts personal information. With Thefacebook.com you had to sign up with a valid Harvard e-mail address, in that way you know that they are actual people, and mostly students. This made it easier to post more personal information like cell-phone number, home address and even sexual orientation. The concern was not about security, but more about wasting time, it became an addictive pleasure. 

\paragraph{}
It didn't take long before Mark Zuckerberg began to receive e-mails from other colleges, requesting to get Thefacebook at their schools. The site was easily scalable, the concern rather laid in how to maintain the intimacy and the clubby appeal. When Thefacebook expanded to the colleges Colombia, Yale and Stanford, students were only able to search and see people from their respective college. Only with permission from a student from another college could you add the person to your friend list. This is a key factor to Facebook's success. Zuckerberg wanted people to post personal information and create a more open school community.

\paragraph{}
In June 2004, when the school year was over, Thefacebook had expanded to over forty schools, with 150 000 users. With the rapid expansion, the need for investors and more capacity increased. Zuckerburg moved his base to California and removed the "the" from the name. Thefacebook became just Facebook.

October 2005 Facebook expanded to universities in England, Mexico and Puerto Rico, and in September 2005 a high school version was available \cite{FacebookHistory}. This was a big step for Facebook. All high school members needed an invitation to be able to join. Zuckerburg launched the possibility for all users to see the profiles and send friend request to everyone in the network, the older users had strong objections. College students did not like the idea of high school kids looking at their profiles and being able to befriend them. But with the rapid expansion Facebook was forced to make the site more open and knock down some of the walls dividing the users. Facebook made it possible for employees at different companies like Apple and Microsoft to join the network. 
At the end of 2005 Facebook was used at over 2000 colleges and at over 25 000 high schools in United states, Canada, Mexico, England, Australia, New Zealand and Ireland. 

Up to this point you had to be a student at a college of high school, or employee at a certain company to be able to join the network. After September 2006 everyone over the age of thirteen, with an valid e-mail address, could join. The site was no longer restricted to schools and was now open to the whole world. 

\paragraph{}
By 2009 Facebook had 200 million active users, and was finally getting more users than Myspace, becoming the world's biggest social network \cite{FacebookStoryInceptionToIsp}. With the release of iPhone in 2007, and the launch of Facebook's mobile application in 2008 a new way of sharing became reality. The mobile application enabled Facebook users to send pictures, status updates and comments in real-time. Facebook introduced the "like" button in 2010, together with the growing application and gaming platform. 

\paragraph{}
The movie "The Social Network" directed by David Fincher and Aaron Sorkin came out in October 2010. It is an american drama movie based on the early days of Facebook's history. The popular movie has received many awards, among them 3 oscars \cite{TheSocialNetwork}. 

\paragraph{}
The Facebook timeline was introduced in December 2011 \cite{EvolutionOfFacebook}. The new interface makes the entire history of the users visible, all photos, links, pages you have liked, comments and other things that you have shared on Facebook. 

\paragraph{}
In April 2012 Facebook announces that they are buying the photo sharing application Instagram for \$1 billion . This was the biggest acquisition that Facebook has done \cite{FacebookInstragram}. Instagram just finished a great year with the launching of the android application and a huge growth, with more than 30 million users, and more than five million pictures being uploaded every day \cite{BBCFacebookGrowth}. 
Just a month later Facebook goes public, another big step for Facebook. Each stock were sold for \$38 dollas, giving the company a market value of \$104,2 billion dollars, becoming the highest valued company in history. Facebook's market value was almost 4 times higher than Google in 2004 \cite{EvolutionOfFacebook}. 
  
  
\paragraph{Facebook today}
As of September 2013 Facebook have 5 794 employees divided on 13 offices in the United States, and 24 international offices \cite{keyFacts}. Worldwide Facebook have 1,19 billion monthly active users. About 80\% of the daily active users (727 millions) are from outside of the U.S and Canada. The mobile platform that Facebook presented in 


- Graph search

- Hashtags

\section{Facebook Privacy}
\label{sec:relatedwork_facebookprivacy}
There exists numerous articles and papers written on the development of Facebook privacy, and many researchers have tried to map the human behaviour in regard to Facebook through for example the use of surveys. One of these articles is "Facebook privacy settings; Who cares?" by danah boyd and Eszter Hargittai \cite{whocares}. The paper addresses a survey conducted on a cohort of 18- and 19-year-olds in 2009 and in 2010. The survey focused on their attitude and practice when it came to Facebook privacy settings. During this period, between 2009 and 2010, Facebook made many changes to their privacy settings. This was a turbulent period in Facebook history, with a lot of attention in media.

\paragraph{}
The demographics collected in the survey described in the paper by boyd and Hargittai was sex, age, race and ethnicity and parents' highest level of education. The ladder was used as a "measure" for socio-economic status. The demographics showed a diversity in the people taking the survey. The other data collected consisted of information within these topics: "Internet experience", "Use of Facebook", "Engagement in certain activities on social network sites among Facebook users" and "Experience with Facebook's privacy settings". Based on their discussion and conclusion, we have highlighted some of their findings: 

\begin{enumerate}
\item Majority of young adults using Facebook have to some degree checked their privacy settings. Number of people who had checked increased from 2009 to 2010. One reason for this may have been the media attention Facebook received as mentioned above. 
\item How familiar someone is with technology plays a role in how they handle their Facebook privacy settings. The reason for this assumption is withdrawn from the relationship between changing privacy settings and the frequency of Facebook use, as well as Internet skill. Considering the default settings, this is especially important since the least skilled people get more vulnerable when Facebook changes the default privacy settings. 
\item Among the majority both genders are equally confident in changing their Facebook privacy settings. 
\end{enumerate}

danah boyd and Eszther Hargittai concludes, based on their findings, that experience and Internet skill is important to take into account in regard to how people handle their privacy settings on Facebook. It is incorrect to think that the Facebook users have the same approach to the site. This kind of thinking leaves a part of the users more exposed. It is therefore very important that the people who configure the default privacy settings take these users into consideration. They should be aware of the fact that every user is different and have a different basis of understanding. 

Another relevant article on the topic of Facebook privacy settings is "Analyzing Facebook Privacy Settings: User Expectations vs. Reality" \cite{expectations}. It addresses to what degree the Facebook privacy settings match the expectations of the users. To find information about the users view on the topic, they conducted a survey via Facebook with people recruited from Amazon Mechanical Turk (More information on Amazon Mechanical Turk can be found in section \ref{sec:amt}). They got 200 users who completed the survey. The average values for the users were: 248 friends, 363 uploaded photos, 185 status updates, 66 links, 3 notes, 2 vidoes. Their analysis is centred around two questions, and one of these questions are interesting for us to look at: \emph{"What are the ideal privacy settings desired by users? How close are these to the actual settings that users have?"}

They had some very interesting results on their survey, and these are the ones we wanted to highlight:
\begin{enumerate}
\item Facebook privacy settings match the users expectations 37\% of the time, and then the settings are not as expected they are almost always more open, and exposes the content to a wider audience than desired.  
\item Modified privacy settings match the users expectations only 39\% of the time. This implies that even though you are aware of you privacy settings, you can still have problems configuring them correctly and as desired.
\item Nearly half of the content shared by the users are shared with all Facebook users. This was desired 20\% of the time. 
\item When the privacy settings on photos have been changed by the user, the privacy settings on these photos match the users expectations less than 40\% of the time. 
\end{enumerate}

As mentioned before there exists much material on the topic of Facebook privacy. We chose to shed light on these two articles, because of the similarities in topic to our paper. Later in our paper we will see if we can draw comparisons between the results in these two articles and our own survey result.

\section{Interdependent Privacy}
\label{sec:intpriv}
In today's society Internet is no longer a privilege, it is a human right. With the evolvement of the online social networks (OSN) the incentive to share personal information has grown drastically. People create profiles at different OSNs and share personal information, pictures and comments with each other. With the enormous data sharing privacy concerns arise. The privacy of an individual users i bound to be affected by the decisions of others. and therefore out of the individuals control. This phenonomen lays the basis for the term \textit{interdependent privacy} \cite{InterdependetPriv}. 

\paragraph{Privacy}
Roger Clarke defines privacy as \textit{the interest that individuals have in sustaining a 'personal space', free from interference by other people and organisations} \cite{privacy}. Further Clarke divides privacy into multiple levels; bodily, personal behaviour, information privacy. Bodily privacy is concerned with the integrity of an individual's body, such as blood transfusion without consent, compulsory immunisation and compulsory sterilisation. Personal behaviour privacy relates to all aspects of behaviour, like sexual preferences, and political and religious actions. Information privacy is a collective term including personal communication and privacy of personal data. These include the ability to communicate, using the desired media without being monitored by others, and claim that data about themselves not automatically should be available to others, even when there is data that should be processed by others.

In this article our focus will be on online privacy, the level of privacy and security of personal data published on the internet. For a user the privacy and anonymity is the most important factor in consideration when using online services. It is a hot topic and now more important than ever, especially when the consequences are unforeseen, and the extent of them are often hard to predict. Biczók and China defines online privacy risks with the basis in Clark's privacy definition as described in the list below \cite{InterdependetPriv};

\begin{itemize} 
\item Personal: Potential loss of information about a user and his/hers behavioural data. This can be done by phishing, hacking to steal secure and sensitive user data, like passwords and pin codes.
\item Relational: Revelation of how a user relate to and communicate with others. Spyware is an offline application that can obtain a users data without the consent of the user. 
\item Spatial: Invasion of  the virtual space of an online user. An example of this can be unwanted comments and posts on a user's blog or social networking page.
\end{itemize}

\paragraph{Social networking privacy}
epic.org/privacy/socialnet

\paragraph{Interdependent privacy}
In today's interconnected world, we share enormous amounts of data every single day. Protecting personal, relational and spatial privacy of individuals is no longer just dependant of only your individual actions, but increasingly depending on the actions of others \cite{InterdependetPriv}. With the continuous growing use of social networks, data sharing has become very easy. We share photos, comments, videos, and links. This increasing data sharing arises the concerns for interdependent privacy. 
 
An example can be if Alice posts and tags a picture of her Facebook friend Bob. Alice finds the picture of Bob funny and sees no problem in posting it. Bob on the other hand, does not share Alice's opinion, he finds the picture embarrassing and inappropriate. Bob wants the picture removed, but by the time Alice comes around to remove it, people have already seen it, and maybe reposted it. Bob's privacy was dependent on what Alice did, and out of his own control. 
 
Sharing information without consent from the users can lead to the emergence of externalities. In economics externalities is defined as the unintended costs or benefits that are imposed on unsuspecting people and that results from economic activity initiated by others \cite{externalityDef}. When the effect is beneficial it is considered a positive externality. A negative externality is when the side-effect is negative. Let us relate this to our example with Alice and Bob. When Alice shares the photo without Bob's consent, it might be at benefit for Alice (in personolized experience), but for Bob it will be received as a negative externality, a loss of his online privacy.  
Another example of interdependent privacy is the Facebook platform for third-party applications (apps). How your privacy depend not only on your actions, but also on the actions of your friends. We will discuss this in more detail below.   

\paragraph{}
(Her skal vi skrive kort om Facebook sin app-platform)

\paragraph{}
The article "Third-Party Apps on Facebook: Privacy and the Illusion of Control" was written in the end of 2011 and looks at the privacy threats with the use of third-party apps on Facebook \cite{thirdPartyApps}. In this paper the authors look at what information the third-party applications request when you install them, and how easy it is for an application to retrieve more information from a user than what the user initially want to. There has not been done any other studies on this topic before the time this article was written. Their aim is to increase user control of the apps' data control and alert the users when the apps' violate your initial privacy setting. When a user wants to add an application, the application is required to ask for permission to access certain information, like your "basic information", which includes name, profile picture, gender, networks, list of friends and other information that a user has publicly available to everyone. Other permissions that apps frequently ask for is "post to my wall", "send me email" and "access my profile information". You can later go to your settings and change what information you share with the apps. But by this time you may already have shared information that you initially wanted to keep private. As an example say that a user, we call her Alice, would like to keep her birthday private and have stated this in her privacy settings. Alice then install an app called "Happy Calendar", that let her keep track of friends' birthdays. When installing the app, they asked for permission to access hers and her friends' birthdays in addition to her basic information. Alice allows the app premission, to later find out that "Happy Calendar" has created an album with a calendar image showing the profile pictures to all her friends including herself. This album was posted on her wall and Alice's friends receives notifications about the album. The birthday that Alice initially wanted to keep private is no longer private. The article states that there should be more evident to the user when the app ask for information that is in conflict with the user's privacy settings. In the article two new designs of the approval page are presented and tested.  From the tests it was clear that users was not always aware of what they share, and that a more extensive and informative permission-page would be necessary. It is important that the users understand what they are sharing and that apps often ask for information that you do not want others to see.


\section{Amazon Mechanical Turk}\label{sec:amt}
\paragraph{}
The growth of the Internet have made it easier to conduct studies, surveys and so on. One commonly used technique for conducting these studies and surveys are called \emph{crowdsourcing}. Crowdsourcing is a technique where you outsource a job to a undefined group of people. The beneficial aspects with crowdsourcing is that you are provided access to a large set of people who are willing to do the tasks you want done, for low pay \cite{AMT}.

\paragraph{}
Amazon Mechanical Turk is a good example of a crowdsourcing site. Amazon Mechanical Turk is a Internet marketplace where human intelligence is utilized to perform various tasks \cite{amazonweb}. The people using Mechanical Turk are separated into two groups. You have the \emph{requesters} that post jobs/tasks, and the \emph{workers} who can choose from these jobs/tasks, and execute them for pay \cite{AMT}. The jobs are posted as HITs (Human Intelligence Tasks). HITs are individual tasks that workers can complete to make money. 

\paragraph{The Turk}
The name "Mechanical Turk" comes from a chess-playing automaton from the  late 18th century. The Turk, as it was called, was a construction made to seem like a automatic chess-playing machine. In reality there was a chess-pro inside the machine, that steered the arms of the doll that was on the other side of the chess-board. The Turk was constructed in 1770 by the Austro-Hungarian Wolfgang von Kempelen. The reason for this construction was that von Kempelen wanted to impress the Empress Maria Theresia of Austria. 
The Turk toured around Europe and in America for decades, without anyone knowing the secret of the machine. The chess-pro that operated the construction played and defeated many, including Benjamin Franklin and Napoleon Bonaparte. Although many suspected that the Turk was steered by a hidden human, the trick was not exposed before 1820. The Turk was ruined in a fire in 1854 \cite{theturk}.

\begin{figure}[h!]
\centering
\includegraphics[width=0.7\textwidth]{Turk-open.jpg}
\caption[Engraving of the Turk]{\textbf{Engraving of the Turk.} This figure shows the Turk with open doors and the different parts inside of the Turk. Wolfgang von Kempelen may have drawn this picture himself, since he was a talented engraver \cite{theturk}.}
\end{figure}

\paragraph{Advantages with Amazon Mechanical Turk}
There are several advantages of using AMT for conducting behavioural research surveys. Amazon Mechanical Turk enables the opportunity to reach out to a wide audience, since it provides access to a large subject pool \cite{AMT}. When conducting a survey or other research for example in connection with school projects etc., you seldom have access to a large subject pool. Usually you may get your friends to contribute, and maybe some other people going to the same school or a few people living in the same place. The results of this survey or research will most likely be reflected by lack of diversity. If you use Amazon Mechanical Turk instead, you get yet another advantage; subject pool diversity. The workers on Mechanical Turk are spread all over the world, and have different backgrounds. They have different religions, ethnicity, languages, different positions in society (economical), and age. The one last advantage with Mechanical Turk worth mentioning is that you get access to all the aspects mentioned above at a low cost. The workers are willing to take jobs and perform task for relatively low pay \cite{AMT}.

\paragraph{Financial Incentives} 
Some concerns regarding the financial incentives are brought up in connection with Mechanical Turk (MTurk). One question is whether or not lower pay result in lower quality in the work conducted by the workers. It is important to have knowledge about the relationship between how good the workers perform, and the financial incentives given to them \cite{incentivesAmt}. Research done by Horton and Chilton \cite{amtpay} shows that the least amount of pay a worker is willing to accept for a task on MTurk is \$1.38 per hour, and they refer to this amount as the \emph{reservation wage}. 
\subparagraph{}
The article "Analyzing the Amazon Mechanical Turk Marketplace" \cite{averagepay} written by Panagiotis G. Ipeirotis in December 2010 shows that the effective hourly wage on MTurk is \$4.80. This is calculated based on some observations, and also on some assumptions. What they observed was that the median arrival rate was \$1.040 per day, and that the median completion rate was \$1.155 per day. They then assumed that MTurk acts like an M/M/1 queuing system. Based on these observations and assumptions they used basic queuing theory and calculated that a task worth \$1 is completed with an average of 12.5 minutes. Like mentioned earlier, this results in an effective hourly wage of \$4.80.
\subparagraph{}
Winter and Mason \cite{incentivesAmt} conclude that if you increase the pay, the quantity of participants increases, but the quality of the work done does not increase. They think the reason for this is the \emph{anchoring effect}. The anchoring effect describes that it is common for humans to depend too much on the first information given to them when making decisions \cite{anchoring}. In the case Winter and Mason presents: the workers who get more pay, also assume that the work they are about to conduct is more extensive, and therefore do not get more motivated to perform the work. 

\section{SurveyMonkey}
\label{sec:sm}
SurveyMonkey is the world's leading provider of web-based survey solutions \cite{surveymonkeyaboutus}. SurveyMonkey was founded in 1999 by Ryan Finley, and had 15 million users in 2013 \cite{surveymonkeywiki}. Using SurveyMonkey as a tool you are allowed to create your own survey based on templates. To get started with SurveyMonkey and to create surveys you have to register the site, and choose account type after need. The different account types have different prices. The more expensive, the more is included. There are several features available when using SurveyMonkey \cite{surveymonkeyfeatures}. It is easy to create questions, with 15 question types available. You can also add logic to the questions. It is easy to customize the appearance of the survey, with the colors you prefer and so on. Getting responses on the survey is done by sharing an URL, for example on Facebook or in e-mails. When you have gotten answers on your survey, you get the data presented in graphs and charts. You can also export the results in various ways, for example all response data or just individual responses.

