\chapter{Related Work}
\label{chp:relatedwork} 


\section{Online Social Networks}
\paragraph{}
Since Facebook was introduced to the public in 2006, it has grown to be the largest online social network (OSN) in the world. OSNs have a peer-to-peer architecture, and therefore makes it easy for members to initiate communication with whom they want, given that they are also connected to the network. OSNs also enables the possibility for people to easily publish and retrieve information about subjects of interest \cite{DPBook}. The growth of Facebook has made it necessary to introduce new ways to manage privacy and ensure a secure online environment. When it comes to OSNs the privacy embedded in the program/app etc. is not enough to ensure such an environment, due to the interdependent privacy issues. Your privacy is to a large extent affected by the privacy decision of others.

\begin{figure}[h!]
\centering
\includegraphics[width=0.2\textwidth]{facebook-icon.png}
\caption{The Facebook Icon}
\end{figure}



\section{Interdependent Privacy}


\section{The History of Facebook}
It all started in 2004 when the Harvard sophomore Mark Zuckerberg and three of his classmates created the webpage facesmash. Zuckerberg hacked into the administrative database to extract the ID photos of all the students. The webpage presented two and two photos creating a “hot or not” game for his fellow students. Within the first hou facesmash had 450 visitors and 22 000 photo-views. \cite{FacebookHistory}.


\section{Amazon Mechanical Turk}


