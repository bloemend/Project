\chapter{Default Privacy Settings}
\label{chp:defaultprivacysettings} 

\section{Default Privacy Settings}\label{sec:first_section}
In this chapter we are going to look into the history of Facebook’s privacy settings, and map the development from 2006 until today. 

\paragraph{}
Facebook has evolved from being a networking site for students attending Harvard to becoming a global phenomenon. Facebook's user interface has gone through several changes over the years, which has brought both joy and frustration to the users. When these changes have been made, there has also been adjustments to the default privacy settings as well \cite{EvoPriv2}. At the beginning, in 2005, when Facebook first was applied outside of Harvard University, the users personal information was only accessible to a users Facebook friends and to people connected to the same network on Facebook \cite{EvoPriv}. 

The changes to the default privacy settings are emphasized in \tref{tab:dps}. 

\begin{center}
\begin{table}
\caption{\label{tab:dps}Changes in the default privacy settings on Facebook from 2005 until today. \cite{EvoPriv,PrivTimeline}}
    \begin{tabular}{ | l | p{9cm} |}
    \hline
    \textbf{Year} & \textbf{Default Privacy Settings} \\ 
    \hline
    2005 & Personal information (e.g., name and profile picture) is 	only visible to specific groups specified in your privacy 			settings.\\ 
    \hline
    2006 & The only information displayed in your profile is your 		school and specified local area. \\ 
    \hline
    2007 & Name, name of school (network) and profile picture 			(thumbnail) is available to all Facebook users.\\
    \hline
    November 2009 & Name, profile picture and demographics is 			available and searchable to the entire Internet. In addition to 	this, list of friends are visible to all Facebook users.\\
	\hline
    December 2009 & Your name, profile picture, list of friends, 		pages you are fan of, demographics and likes are available for 		the entire Internet.\\
    \hline
    April 2010 & The entire Internet can see everything, except 		wall posts that are limited to friends and photos that are 			limited to your network. \\
    \hline
    2011 &  \\
    \hline
    2012 & \\
    \hline
    2013 & \\
    \hline
    \end{tabular}
   \end{table}
\end{center}



\markboth{}{}


\section{Facebook privacy}\label{}
\paragraph{}
Zuckerberg said this in a one of his meetings; “I mean, one way to look at the goal of the site is to increase people’s understanding of the world around them, to increase their information supply,” he said. “The way you do that best is by having people share as much information as they are comfortable with. The way you make people comfortable is by giving them control over exactly who can see what” \cite{MeMedia}.


\paragraph{}
User control became a hot topic i 2006. There had been reported that sex- offenders was using social networks to pick out their victims. MySpace found out that several teenagers had been assaulted by people they meet at the site. Facebook also received some negative mention in the press. Numerous times the campus police had to shut down big parties announced on Facebook. In 2005 a student a Fisher college was expelled after posting this comment about the schools police officer "needs to be eliminated" \cite{MeMedia}.  
In general the privacy settings and restrictions that Facebook has have protected the users. They can easily change the setting and decide who can see what.

“I think that understanding that there might not be any difference between what people are doing online and offline is something really important,” Zuckerberg said firmly. “People are online because it is a more efficient way of doing things.” A man in the audience asked Zuckerberg about some pictures of students drinking at an East Coast college, which, he claimed, had appeared on Facebook and had led to the expulsion of several students. “First of all, it’s pretty stupid if you put up pictures of you doing drugs on Facebook,” Zuckerberg said. “I think that that’s just sort of the deviant behavior on the very far end of the distribution. . . . I bet that those kids do not post pictures of them doing drugs on Facebook anymore.” He went on, “Obviously that’s a pretty shitty way to learn that, like, you’re not supposed to post pictures like that on Facebook, but, I mean, the fact that everyone here hears this and is kind of shocked means that more than just those few people learn from that mistake, right? And the system is going to reach an equilibrium that makes sense” \cite{MeMedia}.

\paragraph{}
"Somehow we missed this point with News Feed and Mini-Feed and we didn't build in the proper privacy controls right away. This was a big mistake on our part, and I'm sorry for it." \cite{FacebookStoryInceptionToIsp}


\section{Facebook Features}\label{sec:facebook_features}

\subsection{Beacon}
-Det må fylles ut mer her!

\paragraph{}
At the end of 2007 Facebook lauched the feature called Beacon. Beacon was created to help users easily share their information with their friends. Beacon was part of and advertisement system that Facebook used. It tracked peoples activity on the web and made it possible for users to share this with their friends. When Beacon was launched it had 44 partner sites, and these sites had the possibility to post a users activity on the user's newsfeed \cite{BeaconWebsites}. Beacon was a sentral element in Facebook's ad system created to connect businesses with users, and making it easier for the businesses to target the advertising. 

\paragraph{}
In a blog post, Mark Zuckerberg apologized for the way the feature was created and for the handling of the complaints in hindsight. 
\cite{Beacon} 
