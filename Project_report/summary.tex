\renewcommand{\abstractname}{Abstract}
\begin{abstract}
With the evolvement of the online social networks, the incentive to share personal information has grown drastically. Along with the enormous data sharing that exists in todays interconnected world, privacy concerns arise. The privacy of an individual is bound to be affected by the decisions of others, and are therefore to some degree out of the users control. This phenomenon lays the basis for the term \textit{interdependent privacy}. In this study we will direct our focus to Facebook, todays larges online social network.  

Interdependent privacy is one specific part of the privacy issues that exists on Facebook. In order to get a full overview and understanding on this matter, we have looked into different aspects. We have mapped the development of the default privacy settings and the most important features introduced by Facebook over the years, and how these features have affected the users' privacy. We have also looked at human awareness in regard to Facebook privacy, how much they care about their privacy as well as their awareness regarding apps' permission requests.

To map human awareness, we constructed a survey. In order to get a good image of people's awareness we distributed the survey on Amazon Mechanical Turk (AMT). This is a marketplace for work that requires human intelligence. One of the key benefits of using AMT is that it provides one of the largest subject pools, with both diversity and low cost. The survey was available on AMT for 3 weeks, and got a total of 250 responses from 13 different countries. We analyzed these results with focus on awareness of privacy settings (including app settings). We wanted to see if there was a connection between privacy settings, and app settings and awareness of the permissions requested when installing apps. 

Results: What's the answer? Give specific results. 

Conclusion: What are the implications of your answer? summary of the discussion of the results and conclusion 
\end{abstract}