\renewcommand{\abstractname}{Abstract}
\begin{abstract}
With the evolution of online social networks, the incentive to share personal information has grown drastically. Along with the progressive data sharing that exists in today's interconnected world, privacy concerns arise. The privacy of an individual is bound to be affected by the decisions of others, and is therefore, to some degree, out of the user's control. This phenomenon lays the basis for the term \textit{interdependent privacy}. In this study, we will direct our focus to Facebook, today's largest online social network.  

Interdependent privacy is one specific part of the privacy issues that exists on Facebook. In order to get a full overview and understanding of this matter, we have looked into different aspects of Facebook privacy. We have mapped the development of the default privacy settings and the most important features introduced by Facebook over the years. We have also looked at user awareness with regard to Facebook privacy, how much they care about their privacy, as well as their awareness regarding app permission requests.

To map human awareness, we constructed a survey. In order to get an adequate picture of people's awareness, we distributed the survey on Amazon Mechanical Turk (MTurk). This is a marketplace for work that requires human intelligence. One of the key benefits of using MTurk is that it provides one of the largest subject pools available, with both diversity and low cost. We analyzed these results with focus on awareness of privacy settings (including app settings). We wanted to see if there was a connection between privacy settings and app settings, and awareness of the permissions requested when installing apps. 

%Results: What's the answer? Give specific results.
Our results show that all our respondents at some point have changed their privacy settings, as nobody had all their settings set to default. The majority of all the respondents check their settings 3-6 times a year. When we look at the corner-points, the ones that check them frequently and those that have not checked their settings during the last year, there is a clear difference. This gave the basis for two hypotheses that we thoroughly investigate in this report. "People that check their Facebook settings seldom, do not have as private/secure settings as the ones who check their settings often. Also, these people do not have as much knowledge about app permission requests, as the ones who check their Facebook settings often" and "People with many apps have less knowledge about the app permission requests, and less knowledge about the existence of the setting "Apps others use"". 

Our research backs up both hypotheses. A higher percentage of the people checking frequently have changed their settings to a more private/secure option, meaning that they are aware of the settings' existence. These users were both younger (almost 10 years in average) and more active (almost 20\% more active), than the ones who seldom check their settings. The people checking frequently are also to a higher extent aware of the app permission requests. As well as the setting "Apps others use", which to a high extent concerns interdependent privacy. 
We also looked at the awareness of the permission requests for the ones having many apps connected to their Facebook account, versus the ones having just a few. The ones with few apps were more aware of all the requests we presented in our survey, and this group was also more aware of the existence of the setting "Apps others use", in comparison to the ones who had many apps connected to their Facebook account. 

An interesting observation was that the respondents of our survey cared more about what they post (comments, photos, etc.) about others, than what is posted about themselves. The knowledge of the term interdependent privacy, as well as the knowledge about the app permission requests, was low.

%Conclusion: What are the implications of your answer? summary of the discussion of the results and conclusion 
The results from our research shows that our initial assumptions were correct. Specifically there exists little knowledge around the issues regarding interdependent privacy. Online privacy is a hot topic in the media these days and people do care about what is posted about them; but still they have poor security/privacy settings and in general little to no knowledge about app permissions. Is Facebook trying to hide the fact that information about you may be shared without your knowledge? If so, is this in conflict with their vision of an open and interconnected world?

\end{abstract}