\chapter{Introduction}
\label{chp:introduction} 

\section{Motivation}

\section{Problem Description}

\section{Methodology}
\label{sec:methodology}
Our assignment is divided into parts, where one consists of collecting data from Facebook users regarding their view on Facebook privacy settings, and the other is a theoretical research of how Facebook privacy settings has evolved since the introduction of Facebook. This means that we have used different approaches to be able to retrieve the information desired.

\paragraph{Approach}
In this section we will describe our approach of collecting data from Facebook users regarding their view on Facebook privacy settings. Facebook is a global social network, so to be able to get more accurate information it is important to reach out to a wide and diverse audience. We decided to use Amazon Mechanical Turk for this purpose. To gather the data, we made a survey for the users to answer. Survey is a common used research method that involves the use of standardized questionnaires or interviews to collect data about people and their preferences, thoughts and behaviours in a systematic manner \cite{survey}. Survey, as a research method, has several advantages in comparison to other methods of doing research. Survey is a good method of retrieving unobservable data, like for example peoples attitudes, behaviours, characteristics, preferences, and demographics. Surveys are great when you want to cover a large group of people, like a country, that otherwise would be difficult to observe. With large groups and large amounts of data, surveys allows small effects to be detected, and makes it easy to compare the subgroups that may appear. Survey are in an economically sense cost effective. It is a lot cheaper for a researcher to make and send out a survey than to use other methods like experimental research. Survey as a research method also has some disadvantages. The method is often exposed to biases, like sampling bias, non-response bias, and social desirability bias. Surveys have a reputation for low responses, hence the non-response bias. This was one of the reasons for choosing AMT as a platform for publishing your survey. 

We started by implementing the survey in Amazon Mechanical Turk (AMT), but learned that the templates provided by AMT was missing some of the features we wished to include, like dividing the questions into several pages. So mainly for design purposes we chose to create our survey in SurveyMonkey. This is easily integrated with AMT and a often used option.  Using SurveyMonkey also made it a lot easier to keep track of answers and see summaries. SurveyMonkey has a great and easily understandable user interface, and made it easy to share the survey to other mediums like Facebook, to reach out to an even larger and more diverse audience. 

\paragraph{}
In AMT we set the requirements that the users had to be "Master Workers". This is users that through a good reputation has earned the title, and by setting this requirement we rule out unserious users and answers. This will save us a lot of time in the screening process. When a user chooses to take our survey, they first get some information about the purpose and incentive of the survey, and a link to SurveyMonkey to take the survey. When the survey is finished the user receives a code that they have to provide before submitting their HIT in AMT. This is an assurance for us that all users on AMT has finished the survey before they get paid. Throughout the lifetime of our survey we have changed this code, just to make sure that nobody tries to get paid without actually doing the work. When the survey is completed in a serious manner the workers get paid \$1,5. On average, the users spent 13 minutes and 37 seconds to take the survey, this gives an effective hourly rate of \$6,61.     
