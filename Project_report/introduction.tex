\chapter{Introduction}
\label{chp:introduction} 

\section{Motivation}
Today's largest online social network, Facebook, has increased drastically in popularity since it was introduced, and people use it to share enormous amounts of information. During the lifetime of Facebook there has been a significant change in the default privacy and security settings. At the same time Facebook's features have been significantly upgraded (e.g., Apps), and the platform itself has expanded to several different platforms (e.g., iOS and Android). Owing to this development, the complexity of privacy-related issues has made the originally embedded privacy requirements inadequate. This makes it interesting to look into privacy issues related to Facebook, and find out how aware people are of the existence of the various settings. Facebook has, for many, also become a sort of "snooping"-tool and it is therefore important that the users are able to protect the information they do not want to share with the public.

Interdependency is a reciprocal relation between two or more decision-making entities, whose actions have consequences for each other. Interdependency is a very important issue when it comes to social networks, since your privacy is affected by the decisions of others. Applications on Facebook introduced a whole new dimension to the privacy issue, in particular issues regarding interdependent privacy. It is therefore interesting to look at people's awareness when it comes to the information they share with apps, and how apps utilize this information. Previous studies have shown that the apps' permission requests are often ambiguous, and that permissions often goes against the users' privacy settings \cite{thirdPartyApps}.  


\section{Problem Description}
Based on our motivation we did research and conducted a survey in order to find the answers to the following questions: 

\begin{enumerate}
\item What kind of settings does Facebook offer today?
\item How have the default privacy settings on Facebook evolved over time?
\item What are the default settings today?
\item What kind of features have Facebook introduced that at some point have affected the users' privacy?
\item How aware are the users of the settings on Facebook, and how often do they check them?
\item To what degree do people care about what they post about others (photos, comments etc.), and what others post about them? 
\item To what degree are people aware of the different app permission requests, and do they know the meaning of the term interdependent privacy?
\end{enumerate}



\section{Methodology}
\label{sec:methodology}
Our assignment is divided into two main parts. One part consists of collecting data from Facebook users to find their view on different Facebook settings, and to check their awareness about the different app permission requests. The other part is a theoretical study of how Facebook privacy settings have evolved since the beginning of Facebook. We have used different approaches to be able to retrieve the information we needed.

\paragraph{}
Facebook is a global social network, so to be able to get more accurate information it is important to reach out to a wide and diverse audience. We decided to use Amazon Mechanical Turk (MTurk) for this purpose. To gather the data, we made a survey for the users to answer. Survey is a commonly used research method that involves the use of standardized questionnaires, or interviews, to collect data about people and their preferences, thoughts and behaviours in a systematic manner \cite{survey}. Survey, as a research method, has several advantages in comparison to other methods of doing research. Survey is a good method of retrieving unobservable data, like for example peoples attitudes, behaviours, characteristics, preferences, and demographics. Surveys are great when you want to cover a large group of people, like a country, that otherwise would be difficult to observe. With large groups and large amounts of data, surveys allow small effects to be detected, and they make it easy to compare the subgroups that may appear. Survey are cost effective in an economic term. It is a lot cheaper for a researcher to make and send out a survey, than to use other methods like experimental research. Surveys, as a research method, also have some disadvantages. The method is often exposed to biases, like sampling bias, non-response bias, and social desirability bias. Surveys have a reputation for having low responses, hence the non-response bias. This was one of the reasons for choosing MTurk as a platform for publishing our survey. 

We started by implementing our survey on Amazon Mechanical Turk, using one of their predefined templates. However, we quickly discovered that it was missing some important features we needed to include in our survey, like dividing the questions into several pages. So, mainly for design purposes, we chose to create our survey in SurveyMonkey instead. SurveyMonkey surveys are easily integrated with MTurk, making it a popular tool for MTurk users. Using SurveyMonkey also made it a lot easier to keep track of answers and to see summaries. SurveyMonkey has a great and easily understandable user interface, and they have made it easy to share the survey to other mediums, like Facebook, to reach out to an even larger and more diverse audience. 

\paragraph{}
In MTurk we gave the requirement that the respondents had to be "Master Workers" to be able to take our survey. These users have earned the title by building up a good reputation, and by setting this requirement we ruled out dubious users and answers. This saved us a lot of time in the screening process. When a user choose to take our survey, they first got some information about the purpose and incentive of the survey, and a link to SurveyMonkey. When the survey was finished the user received a code that they had to provide before submitting their HIT on MTurk. This was an assurance for us that all users on MTurk had finished the survey before they got paid. Throughout the lifetime of our survey we changed this code, just to make sure that nobody tried to get paid without actually doing the work. When the survey was completed in a serious manner the workers got paid \$1,5. On average, the users spent 13 minutes and 24 seconds to take the survey, this gives an effective hourly rate of \$6,7. 

\section{Limitations} 
Our main limitation was the amount of time available to finalize our specialization project, as we only had 4 months at our disposal. Our research analysis is based on the 250 responses we got on our survey. This is not enough to get a full and accurate image of the research area, but it gives a solid basis for further work. The amount of money granted for conducting the research was also limited, and this meant that we could not pay an unlimited amount of workers on MTurk. In addition to this, there is not an infinite number of workers available on MTurk. If we were to reach out to even more people, we would need to use several different arenas. We have not implemented a software-tool in order to cover a wider field in our research. Even though we distributed our survey on MTurk and Facebook, it is up to the user whether or not to take the survey. We had no control over the users' intentions. A normal problem with surveys is that there exists no way for us, as researchers, to verify that the respondents have answered in a truthful and proper manner. Another factor to have in mind, is overlooking valuable research data when carrying out analysis. 


