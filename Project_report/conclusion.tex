\chapter{Conclusion}
\label{chp:conclusion} 

Facebook offers a high variety of settings for the users. The users are given full control over their settings, and can themselves choose whom they want to share information with. As default most of the settings are set to the most public option available. Facebook states that the user have full control, and that is the users responsibility to control the audience for the information they share. They also state that the users should not share information they do not want anyone to see, like posting a picture of doing drugs. The most important aspect regarding the Facebook settings is user awareness. If a user is not aware of the existing settings, they are of no value for the user. Many of the respondents expressed that they through taking our survey they became aware of settings they did not know existed. The survey did not only give us valuable research results, but were also informative and educational for the respondents.  

When Facebook started as a online environment only available for Harvard students, the privacy was the most important aspect. You needed a valid Harvard email-address in order to sign up. This was an assurance that all users of "Thefacebook" were actual people, and this made the barrier of sharing information much lower. In 2005 nothing was publicly available to the entire Internet. Most of the information a user shared was restricted to network or friends, expect name, profile picture, gender and network that was available to all Facebook users. The default settings became more and more public for every year that went by. Major changes to the default settings happened in November 2009. Then name, profile picture, gender and networks became available to the entire Internet. Most information (wall posts, photos, likes, birthday) was restricted to friends of friends. There exists a theory called "Six degrees of separation" \cite{six}, which states that everyone and everything is six or fewer steps away. This means that two people are connected via maximum six steps. So when information is shared with friends of friends, the audience is extremely large. Big changes to the default settings also happened in April 2010, when everything, except contact info and birthday, became publicly available to the entire Internet. With the information becoming increasingly available on the Internet, Facebook also announced more settings possible for the users to edit. Today the entire Internet can see everything by default, except posts you have been tagged in on your Timeline, and others posts on your Timeline which are limited to friends of friends. There is no doubt that Facebook's development of default settings have become increasingly public. With this development the importance of the users awareness of the existing settings also becomes more and more crucial. It is important that the users follow the development, and change the settings according to their personal preferences. 

Along with the development of Facebook, the site also introduced numerous of new features. Some of them affected the users privacy more than others, and received a lot of attention in media. The feature that had the most impact on users' privacy was Facebook's introduction of applications. 
The main reason that apps affects a users privacy is because of the information apps retrieve. What kind of information apps requests to access is very often unclear to the user. When a user install an app on Facebook, the application shows the user a list of permissions they request to access. These permissions often expand beyond the basic information, such as asking to post on a users behalf, and sometimes asks to access relational information (this may include private chat messages). Often these requests asks for permissions that strides against the privacy settings, and leads to a user sharing information that they originally wanted to keep private. Our research shows that there have been a change in how the permissions request are shown to the user. In 2011 there was a separate page displaying the permission requests. Today Facebook have introduced the App Center, where the permissions requests are shown in small letters on the side of the page, not very obvious to the user. Research done on the permissions page in 2011 showed that it should be more clear to the user when they approved permission request that stride against their initial privacy settings. When we look at the permission page today, Facebook have done the opposite. There is less focus on the permissions, and more on the app in general.  
There is no doubt that Facebook has become a more public and open platform, and work on the idea of sharing. Facebook always want to customize all information, adds, apps and Newsfeed posts, to the users based on their preferences, like friends interests etc. 

Users' awareness when it comes to the settings available on Facebook is extremely important. The majority of all the respondents that took our survey stated that they check their Facebook settings to some degree. Most of them, almost half of the respondents,  stated that they check their settings 3-6 times a year. On the contrary only 12\% stated that they had not checked their settings during the last year, but they must have done it at some point since their settings differ from the default setting. We saw a contrast in the ones that seldom check their settings and the ones that frequently check. Initially we 
thought that the ones that frequently check their Facebook settings, have more knowledge of the settings that exits, both the privacy/security settings and the settings regarding apps, as wells as being more active users. This assumption turned out to be correct. The difference between the ones that check seldom and the ones that check frequently were not immense, but large enough to divide these groups. The ones that frequently check had more secure settings on all aspects. These users were in average almost 10 years younger than the ones seldom checking. This might be because of the technological world that today's young generation have grown up in. The elderly generation does not have the same technological knowledge. They are on Facebook because they have understood that in todays world that is how they stay in contact with old friends and relatives. Several of our respondents express that they have created a Facebook profile of the single reason that to stay in contact with their younger family members. 

The ones checking their settings frequently also were more aware of the permissions apps request, than the ones seldom checking. There was also a difference when we looked at the ones having many apps and the ones having few apps connected to their Facebook. We assumed that the ones having few apps connected had more knowledge of the the permissions requests. When knowing how much information an app can access, one would be careful with connecting apps to Facebook. The amount that was aware of the permissions apps request was higher for the ones having few apps connected to their Facebook. Even though it was higher knowledge for the ones having few apps than having many, the overall knowledge of these permissions were low. We think many knows that the applications may access information and therefore refrain from installing any apps, in other words just ignores everything regarding apps. But what one might forget is that even though one might refrain from apps, their friends might not share the same view. For example if Alice chooses to refrain from everything regarding apps, Alice's Facebook friend might love apps and might not be aware that he many times allows apps to access information about Alice, totally without Alice's knowledge. Here the term interdependent privacy really is brought in to light. Alice's privacy is dependent on what Bob does, and to some extent out of her control.   
The individual user can choose what information they choose to share with apps their feinds use, this is done in the setting "Apps others use". When it came to the awareness of this setting , as many as 70\% of the total respondents were unaware of it's existence. In other words 70\% of our respondents have shared all types of information with apps their friends install, without being aware of it.

<<<<<<< HEAD
There is also a difference in how much people care about what they post about others as well as what is posted about them. An interesting observation was that people in general care more about what they post about others than about what is posted about them. Comments from the respondents shows that they are restrictive in what they chose to share about others, and a rule that repeated is that they do not share anything that would not like to be shared of themselves, or that could not be viewed of certain people, like parents, or form a parents perspective, their children. It was also possible to see a difference between men and women, where women care more about the settings rewarding what is posted of them (photos, tags in comments, etc.) and men care more about security settings. 
But how can the respondents care so much about about what they post, as well as what is posted about them, but not know how much information is shared without their knowledge through apps? 
=======
What kind of features have Facebook introduced that at some point have affected the user's privacy?
How aware are the users of the settings on Facebook, and how often do they check them?
To what degree do people care about what they post about others (photos, comments etc.), and what others post about them? 
Is there a difference in what kind of settings people value depending on  demographics?
To what degree are people aware of the different permissions Facebook apps request, and do they know the meaning of interdependent privacy?
>>>>>>> 0a25e1608b523cddc03c7dd27b1adb0f9685e9c4


The term interdependent privacy is relatively new, and a term that is becoming more and more important. When we asked whether or not our respondents knew what is meant by the term in regard to Facebook, a lot of people tried to formulate a definition. It was clear that most did not understand the meaning, and together with the low knowledge of the permissions apps ask for, there is no doubt that this an area that should need more focus and attention. As well as an area where users to an higher extent could be educated. 

\section{Furtue Work}
The research we conducted in this paper have just touched the surface of the privacy related issues on Facebook and interdependent privacy related to the App Center Facebook offers. There are many aspects to look at for future work. The survey and results we have presented can lay the basis for suggestions for improvements for Facebook. One possibility is to conduct a more detailed and extensive survey. An other approach is to direct the focus more on apps and for example create an app. By doing this look at the privacy related issues form the inside and out. The possibilities are many as this is a big and hot topic. 

%Når dere kommer til Conclusion bør dere summere opp resultatene sammen med spørsmålene dere ønsker å besvare, og hvor godt dere mener at dere har klart å besvare dem.
% I tillegg bør Conclusion inneholde forslag til videre arbeid.

%Facebook er en samfunnsting, og at folk er ikke der fordi de er teknologiske av seg, men fordi de føler de "må" for å holde pace. 