\chapter{Conclusion}
\label{chp:conclusion} 

Facebook offers a high variety of settings for the users. The users are given full control over their settings, and can themselves choose whom they want to share information with. As default, most of the settings are set to the most public option available. Facebook states that the user have full control, and that it is the users responsibility to control the audience for the information they share. They also state that the users should not share information they do not want anyone to see, like posting a picture of someone doing drugs. The most important aspect regarding the Facebook settings is user awareness. If a user is not aware of the existing settings, the settings are of no value to the user. Many of the respondents expressed that through taking our survey they became aware of settings they did not know existed. In other words, the survey did not only give us valuable research results, but were also informative and educational for the ones who took it.  

When Facebook started as a online environment only available for Harvard students, the privacy was the most important aspect. A valid Harvard email-address was needed in order to sign up. This was an assurance that all users of "Thefacebook" were actual people, and this made the barrier of sharing information much lower. In 2005 nothing was publicly available to the entire Internet. Most of the information a user shared was restricted to network or friends, expect name, profile picture, gender and network that was available to all Facebook users. The default settings became more and more public for every year that went by. Severe changes was made to the default settings in November 2009. Name, profile picture, gender and networks became available to the entire Internet. Most of the remaining information (wall posts, photos, likes, birthday) was restricted to friends of friends. There exists a theory called "Six degrees of separation" \cite{six}, which states that everyone and everything is six or fewer steps away. This means that two people are connected via maximum six steps. This means that when information is shared with friends of friends, the audience is \textit{extremely} large. Major changes to the default settings also took place in April 2010, when everything, except contact information and birthday, became publicly available to the entire Internet. With the information becoming increasingly available on the Internet, Facebook also announced more settings possible for the users to edit. Today the entire Internet can see everything by default, except posts you have been tagged in on your Timeline, and others posts on your Timeline, which are limited to friends of friends. There is no doubt that Facebook's development of default settings have become increasingly public. With this development the importance of the users awareness of the existing settings also becomes more and more crucial. It is important that the users follow the development, and change the settings according to their personal preferences. 

Along with the development of Facebook, the site also introduced numerous of new features. Some of them affected the users privacy more than others, and received a lot of attention in media. The feature that had the most impact on users' privacy was Facebook's introduction of applications. 
The main reason that apps affects a users privacy is because of the information apps retrieve. What kind of information apps requests to access is very often unclear to the user. When a user install an app on Facebook, the application shows the user a list of permissions they request to access. These permissions often expand beyond the basic information, such as asking to post on a users behalf, and sometimes asks to access relational information (this may include private chat messages). Often these requests asks for permissions that strides against the user's privacy settings, and leads to a user sharing information that was intended to be private. Our research shows that there have been a change in how the permission requests are shown to the user. In 2011 there was a separate page displaying the permission requests. Today Facebook have introduced the App Center, where the permission requests are shown in small letters on the side of the page, not very obvious to the user. Research done on the permissions page in 2011 showed that it should be more clear to the user when they approved permission request that stride against their initial privacy settings. When we look at the permission page today, Facebook have done the opposite. There is less focus on the permissions, and more on the app in general.  
There is no doubt that Facebook has become a more public and open platform, and work on the idea of sharing. Facebook always want to customize all information, adds, apps and Newsfeed posts, to the users based on their preferences, like friends interests etc. 

Users' awareness when it comes to the settings available on Facebook is extremely important. The majority of all the respondents that took our survey stated that they check their Facebook settings to some degree. Most of them, almost half of the respondents, stated that they check their settings 3-6 times a year. On the contrary only 12\% stated that they had not checked their settings during the last year, but they must have done it at some point since their settings differed from the default settings. We saw a contrast in the ones that seldom check their settings and the ones that frequently check their settings. Initially, we 
thought that the ones that frequently check their Facebook settings, have more knowledge of the settings that exits, both the privacy/security settings and the settings regarding apps, as well as being more active users. This assumption turned out to be correct. The difference between the ones that check seldom and the ones that check frequently was not immense, but large enough to divide these groups. The ones that frequently checked had more secure settings on all areas. These users were in average almost 10 years younger than the ones seldom checking. This might be because of the technological world that today's young generation have grown up in. The elderly generation does not have the same technological knowledge. One reason for them being on Facebook is that Facebook provides them the opportunity to stay in touch with old friends and relatives. Several of our respondents express that they have created a Facebook profile of the single reason; stay in touch with their younger family members. 

The ones checking their settings frequently also were more aware of the permissions apps request, than the ones seldom checking. There was also a difference when we looked at the ones having many apps and the ones having few apps connected to their Facebook. We assumed that the ones having few apps connected had more knowledge of the the permission requests. When knowing how much information an app can access, one would think a user would be careful with connecting apps to Facebook. The amount that was aware of the permissions apps request was higher for the ones having few apps connected to their Facebook. Even though it was higher knowledge for the ones having few apps than for the ones having many apps, the overall knowledge of these permissions were low. We think that many users know that the applications may access information and therefore refrain from installing any apps, in other words just ignores everything regarding apps. But what one might forget is that even though one refrain from installing apps, their friends might not share the same view. For example, although Alice choose to refrain from everything regarding apps, Alice's Facebook friend, Bob, might love apps. Bob may not be aware of the fact that he allows his apps to access information about Alice. This is where the term interdependent privacy becomes interesting to look at. Alice's privacy depends on the actions of Bob, and are to some extent out of her control. The individual user can choose what information they want to share with apps their friends use. This is done in the setting "Apps others use". When it came to the awareness of this setting, as many as 70\% of the total respondents were unaware of it's existence. In other words 70\% of our respondents have shared all types of information with apps their friends install, without being aware of it.

There is also a difference in how much people care about what they post about others, as well as what is posted about themselves. An interesting observation was that people in general care more about what they post about others, in comparison to what is posted about themselves. Comments from the respondents shows that they are restrictive when it comes to what they choose to share about others. It seems like most of the respondents do not share things about others that they would not liked shared about themselves, or that could not be viewed of certain people, like parents, or from a parents perspective, their children. It was also possible to see a difference between men and women. Women seem to care more about the settings regarding what is posted of them (photos, tags in comments, etc.), while on the other hand, men care more about the security settings. The big question them becomes; how can the respondents care so much about what they post and what others post of them, but do not have a clue of how much information is shared without their knowledge through apps? 

The term interdependent privacy is relatively new, and a term that is becoming more and more important. When we asked whether or not the respondents knew the meaning of the term in regard to Facebook, many of them tried to formulate a definition. It was clear that most of them did not know the meaning, and together with the low knowledge of the permissions apps ask for, there is no doubt that this is an area that needs more focus and attention. 

\section{Future Work}
The research we conducted in this paper have just touched the surface of the privacy related issues on Facebook and interdependent privacy issues related to the App Center Facebook offers. There are many aspects to take into consideration for future work. The survey and the results we have presented can lay the basis for suggestions of improvements for Facebook. One possibility is to conduct a more detailed and extensive survey. An other approach is to direct the focus more on apps and for example create an app. By doing this, one can look at the privacy related issues from the inside and out. The possibilities are many as this is a big and hot topic. 

%Når dere kommer til Conclusion bør dere summere opp resultatene sammen med spørsmålene dere ønsker å besvare, og hvor godt dere mener at dere har klart å besvare dem.
% I tillegg bør Conclusion inneholde forslag til videre arbeid.

%Facebook er en samfunnsting, og at folk er ikke der fordi de er teknologiske av seg, men fordi de føler de "må" for å holde pace. 