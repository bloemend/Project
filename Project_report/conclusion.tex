\chapter{Conclusion}
\label{chp:conclusion} 

Facebook offers a high variety of settings for the users. The users are given full control over their settings, and can themselves choose whom they want to share information with. As default most of the settings are set to the most public option available. Facebook states that the user have full control, and that is the users responsibility to control the audience for the information they share. They also state that the users should not share information they do not want anyone to see, like posting a picture of doing drugs. The most important aspect regarding the Facebook settings is user awareness. If a user is not aware of the existing settings, they are of no value for the user. Many of the respondents expressed that they through taking our survey they became aware of settings they did not know existed. The survey did not only give us valuable research results, but were also informative and educational for the respondents.  

When Facebook started as a online environment only available for Harvard students, the privacy was the most important aspect. You needed a valid Harvard email-address in order to sign up. This was an assurance that all users of "Thefacebook" were actual people, and this made the barrier of sharing information much lower. In 2005 nothing was publicly available to the entire Internet. Most of the information a user shared was restricted to network or friends, expect name, profile picture, gender and network that was available to all Facebook users. The default settings became more and more public for every year that went by. Major changes to the default settings happened in November 2009. Then name, profile picture, gender and networks became available to the entire Internet. Most information (wall posts, photos, likes, birthday) was restricted to friends of friends. There exists a theory called "Six degrees of separation" \cite{six}, which states that everyone and everything is six or fewer steps away. This means that two people are connected via maximum six steps. So when information is shared with friends of friends, the audience is extremely large. Big changes to the default settings also happened in April 2010, when everything, except contact info and birthday, became publicly available to the entire Internet. With the information becoming increasingly available on the Internet, Facebook also announced more settings possible for the users to edit. Today the entire Internet can see everything by default, except posts you have been tagged in on your Timeline, and others posts on your Timeline which are limited to friends of friends. There is no doubt that Facebook's development of default settings have become increasingly public. With this development the importance of the users awareness of the existing settings also becomes more and more crucial. It is important that the users follow the development, and change the settings according to their personal preferences. Along with the development Facebook also introduced numerous features. Some of them affected the users privacy more than others, and received a lot of attention in media. The feature that had the most impact on users' privacy was Facebook's introduction of applications. 




\item What kind of features have Facebook introduced that at some point have affected the user's privacy?
\item How aware are the users of the settings on Facebook, and how often do they check them?
\item To what degree do people care about what they post about others (photos, comments etc.), and what others post about them? 
\item Is there a difference in what kind of settings people value depending on  demographics?
\item To what degree are people aware of the different permissions Facebook apps request, and do they know the meaning of interdependent privacy?


%Når dere kommer til Conclusion bør dere summere opp resultatene sammen med spørsmålene dere ønsker å besvare, og hvor godt dere mener at dere har klart å besvare dem.
% I tillegg bør Conclusion inneholde forslag til videre arbeid.

%Facebook er en samfunnsting, og at folk er ikke der fordi de er teknologiske av seg, men fordi de føler de "må" for å holde pace. 